\documentclass{beamer}

\usetheme{Madrid}

\title[SKBD] {SKBD: Segment Knotting Boundary Detection For Room Layout Detection}

\author[Pictures]
{Barna Mikler \and Botond Geonczeol \and Dominik Li \and \\ Matyas David \and Balint Boros}

\institute[ELTE]{Eötvös Loránd University \\ Faculty of Informatics}

\date[2024] {November 2024}

\logo{\includegraphics[height=1cm]{images/elte_cimer_szines.eps}}

\begin{document}

\frame{\titlepage}

\begin{frame}
\frametitle{Table of Contents}
\tableofcontents
\end{frame}  

\section{Motivation}
\begin{frame}
\frametitle{Motivation}
\begin{itemize}
    \item Manhattan World Assumption or cuboid shapes.
    \item Meeting the growing demand for flexible and accurate room layout reconstruction in applications
    \item Automating the floor plan generation process to replace labor-intensive and error-prone manual techniques.
    \item Enhancing robustness to occlusions, clutter, and diverse viewpoints for better performance in challenging environments.
    \item Advancing state-of-the-art methods by introducing SKBD
\end{itemize}

\end{frame}

\section{Reconstruction of Non-Cuboid Spaces}
\begin{frame}
\frametitle{Reconstruction of Non-Cuboid Spaces}
Describing the question
\end{frame}

\section{Terminology}
\begin{frame}
\frametitle{Terminology}
Descriptor
Representation
Segment

%Knot - Later probably
%Super segment - Later probably
%Floor map - Can be possibly skipped
\end{frame}

% Skip some of the methodology or merge frames maybe
% I think they are important, but it might be too long
\section{Detection of Distinct Segments}
\begin{frame}
\frametitle{Detection of Distinct Segments}
Methodology
\end{frame}

\section{Association of Segments}
\begin{frame}
\frametitle{Association of Segments}
\begin{itemize}
    \item Calculating interest score for each segments
    \item Choosing segments with the highest score as key points
    \item Associating segments with key points based on association score, forming knots
    \item Repeat until there are enough knots
\end{itemize}
\end{frame}

\section{Connection of Knots}
\begin{frame}
\frametitle{Connection of Knots}
\begin{itemize}
    \item Determining a super segment for each knot
    \item Calculating connection strength between each super segment, creating the CSSM
    \item Normalizing values in rows
\end{itemize}
\end{frame}

\section{Layout Guess and Reconstruction}
\begin{frame}
\frametitle{Layout Guess and Reconstruction}
Methodology
\end{frame}

\section{Overall Results}
\begin{frame}
\frametitle{Overall Results}
Results
\end{frame}

\section{Results on Structured3D and LSUN}
\begin{frame}
\frametitle{Results on Structured3D and LSUN}
Results
\end{frame}

% Split or give more frames?
\section{Discussion}
\begin{frame}
\frametitle{Discussion}
Discussion
\end{frame}

\section{Conclusion}
\begin{frame}
\frametitle{Conclusion}
Conclusion
\end{frame}

\section{Future Work}
\begin{frame}
\frametitle{Future Work}
\begin{itemize}
    \item Exploring advanced descriptors (e.g. DALSM), potentially improving benchmark results.
    \item Incorporating multi-view stereo or Structure-from-Motion (SfM) techniques to achieve (CE) rates below 0.5\%.
    \item Expanding SKBD’s robustness against varying image qualities and environmental conditions. 
    \item Optimize SKBD for simpler cuboid room layouts without sacrificing its generalizability to non-cuboid spaces.
    \item Reducing computational complexity of the CSSM generation process.
    \item Testing SKBD on diverse real-world datasets.
\end{itemize}
\end{frame}

% We could skip this, kinda useless, but the names are kinda funny
\section{Acknowledgments}
\begin{frame}
\frametitle{Acknowledgments}
Special thanks to
\begin{itemize}
    \item European Union's Horizon Europe Research and Innovation Programme
    \item Dr. Philip Miner
    \item Dr. Keve Tevesy
\end{itemize}
\end{frame}

\section{Q\&A}
\begin{frame}
\frametitle{Q\&A}
Q\&A
\end{frame}

\end{document}
