\section{Related Work}
\label{sec:relatedwork}

Several papers focus on the topic of room layout estimation, with multiple
different approaches to the problem. Most existing methods tackle this problem
with strict assumptions, such as modeling the room by a parametric box (cuboid)
or assuming the room has a single-foor single-ceiling construction.

\paragraph{}

\textbf{Layout Estimation Using Geometric Hints} Room Layout Estimation Using
Geometric Hints\cite{8451365} aims to predict surface layout from a single image
by utilizing a deep network that combines textures and geometric hints. 
Ruifeng et al. \cite{8451365} have presented the use of a multi-scale CNN
approach which incorporates a multi-channel FCN. It works by assigning a surface
label of the five possible surfaces to every pixel of the input image, and then
utilizing a heatmap generated using these labels to determine which part of the
image is predicted to belong to which surface. Under the Manhattan world 
assumption\cite{790349} only these five surfaces \{Left, Front, Right, Ceiling,
Ground\} can be visible in an image at the same time.

\paragraph{}

\textbf{Layout Estimation Using Undirected Graph}
Undirected graph representing strategy for general room layout estimation\cite{YAO2023103963}
paper details a process that focuses on the use of undirected graphs. It parameterizes the layout to a graph, where layout vertices are regarded as the graph's vertices. A neural network then processes this graph and gives the layout as the result. This layout consists of the detected vertices and therefore, the visible walls, the floor and ceiling are also detected. This article also builds on the Manhattan world assumption\cite{790349}, therefore the same rules apply here as well. Subsequently, both cuboid and some non-cuboid shaped rooms were succesfully processed, although only limited information was extracted.