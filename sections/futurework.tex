\section{Future Work}
\label{sec:futurework}
To explore all the new possibilities that the SKBD opens and to attempt to further improve its results, the following points still require future work:
\begin{itemize}
  \item Developing an algorithm with similar principles focusing on cuboid rooms.
  \item Explore the use of different desriptors: Using other descriptors e.g. DALSM can result in a more efficent way with better benchmark results to give an accurate measurement of a pair of images.
  \item {Handling Non-Manhattan World Assumptions:} While SKBD works well under the Manhattan World Assumption, many real-world environments do not adhere to this constraint. Developing a different the algorithm with similar principles might handle non-cuboid rooms without relying on specific assumptions about walls and corners would allow for broader application in real-world scenarios.
  \item Incorporating Multi-view Information: SKBD's ability to combine multiple images of a room from different angles already demonstrates clear benefits. Future work could expand this by integrating multi-view stereo or Structure-from-Motion (SfM) techniques to generate more accurate and detailed 3D room layouts from minimal images
\end{itemize}
