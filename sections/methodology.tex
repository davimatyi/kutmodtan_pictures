\section{Methodology}
\label{sec:methodology}

\subsection{Previous Attempts} %Tried methods
At first, we tried to rely solely on descriptors. The first proposed method was a combination of local and global descriptors that describe various features on the images. We used the SIFT, ORB and SURF descriptors that are local descriptors. Then, using global descriptors we tried to tie the identified segments together. These alone however were not sufficient. This method was unable to capture a more significant connection between the separate images and therefore was not able to be used to reconstruct the room itself. An underlying problem of this method was that local and global descriptors capture different types of information, which are not easily associated with one another.

A different method tried was the involvement of a machine learning model with a well-trained CNN. We hoped that there were some hidden connection points that are accurately detecteable using a neural network. This produced some results, however the use of this model was limited only to a set of cases. In these cases the image sets exceeded the minimal amount of images needed to map out a room with no blindspots. Most image sets consisted of images that had many repeating regions in them, which in our study is a redundant set. We aim to process and reconstruct a room with as few images as possible and minimal amount of repeating image regions.   

A purely algorithmic approach was also tried. There were some promising results using graphs and edge detection. Using this method, we detected the edges of all walls with a modified version of Harris-Corner detection. Then, the developed algorithm managed to connect all corners of a non-cuboid shaped room using graph theory. This ended up not fitting our study, since detecting only the edges of a room is not enough to fully detail it. Therefore, the full recostruction of a room was determined to be impossible using this method. This method took some inspiration from \cite{9707088}, as this paper also used a similar method to determine edges of a room.


\subsection{Detection of distinct segments} %Segment detection
For segment detection, we revisited local descriptors. Using these we were able to detect spots in the image that can be described by some feature. For example color or shape of a given object can be detailed using these descriptors. Using different types of local descriptors, the algorithm looks for the following features in an image: Color, shape, texture, orientation and boundaries of a segment. Some regions will only be detected once by one specific descriptor. But since we use multiple type of descriptors, more than one descriptor can find the same segment. This would be a problem, however the algorithm uses this to its advantage. It uses the redundant information to determine the easily detecteable segments. For the sake of simplicity, in this paper a descriptor detailing a segment is called a representation.

After every possible detail is extracted from an image, we run these representations through a filter that is able to separate the segments into a distinct set. An element of the set consists of a segment and its associated representations. The process of organizing the representations into this set happens by measuring the similarity of two representations against a threshold. If that threshold is reached, the similarity is deemed to be too great and therefore only one of the representation of the segment can remain as the main representation (it is proposed that a high chance of similarity means that the two representations detail the same segment). The unused segment representations are also saved in the set, but only one main representation can be the appointed segment of the set.  

This filtering process also assigns a number to each segment in the set. This number is a so called confidence level, that is able to measure how confident the algorithm is in that this segment is a real, easily distinguishable one. A segment's label is increased if a similar segment was found with a different descriptor. In this case, the algorithm assumes that now more than one descriptor found the same segment and therefore it is more likely that this segment is a real one. 


At the end of this process, we are left with a set of distinct segments, represented by different type of descriptors. Because of the heavy use of different type of feature detection and description, we maximize the number of detectable segments in an image set. The next stage of the process is the association of these segments.

\subsection{Association of the segments} %Segment Association

The first task in the association of segments is determining the interesting and important key points. During the detection phase, some information is already saved about the individual segments, which decriptor found it, how high its activation level was and additional information based on the type of descriptor. While merging them into representations, these characterestics are inherited. Representations also have their confidence score. Based on all these factors, an interest score is calculated for each representation. Then the most important representations will be selected as the first key points.

In the following step, each key point gathers the similar representations around themselves. These groups will be called knots in this paper. An association likeliness score is calculated between each key point and each representation. This includes the similarity of characterestics of the two representation and their disctance. The factors that contributed heavily to the intrest score of the key points are weighted more heavily here. Every representation with an association score above a certain threshold is associated with the given key point, it is placed inside its knot. Each representation can be part of multiple knots independently and there is no upper bound for the possible number of association. Therefore, this part of the process can be heavily parallelized to improve the algorithms performance.

These two steps are repreated iteratively. In the following iterations, the number of associations and their strenght are also calculated into the intrest score of the representations. This ensures that the different representations with different characterestics are selected in each iteration. The desired number of key points and the association threshold are both increased between iterations. While this approach ensures that more and smaller group of representations are gathered around the key points in each iteration, there is no direct hiearchy between them, since the new key points gather representations independently from their previous groups. The iteration of the association ends when a certain number of key points are selected.

\begin{algorithm}
\caption{Segment Association}\label{alg:assoc}
\begin{algorithmic}
\While{\texttt{not enough key points are selected}}
    \For{\texttt{each representation}}
        \State \texttt{calculate intrest score}
    \EndFor
    \State \texttt{select desired number of representation with the highest intrest score as key points}
    \For{\texttt{each key point}}
        \For{\texttt{each representation}}
            \State \texttt{calculate association score}
            \If{\texttt{association score > association threshold}}
                \State {\texttt{associate the key point and the representation}}
            \EndIf
        \EndFor
    \EndFor
    \State \texttt{increase association threshold}
    \State \texttt{increase desired number of key points}
\EndWhile
\end{algorithmic}
\end{algorithm}

\subsection{The process of reconstruction} % Process of recontruction


