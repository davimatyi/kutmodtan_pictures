\section{Discussion}
\label{sec:discussion}

The results obtained from the experiments on both the Structured3D and LSUN datasets provide significant insights into the effectiveness of our proposed SKBD method compared to state-of-the-art approaches. One of the most notable findings is the considerable improvement in both Pixel Error (PE) and Corner Error (CE) metrics when utilizing multiple viewpoints. This demonstrates that SKBD is capable of efficiently leveraging multi-view image data, which is a key differentiator from methods that rely on single images.

\subsection{Multi-Viewpoint Advantage}
The results obtained from the Structured3D dataset emphasize the significant advantages of employing a multi-viewpoint strategy in room reconstruction. SKBD's ability to integrate data from multiple angles allows for a more comprehensive understanding of the scene, which is especially beneficial in overcoming challenges that single-view methods often face. For example, occlusions where certain parts of the room are hidden from view—or clutter, which obscures crucial features, typically introduce inaccuracies in methods like MC-FCN and HGC. However, SKBD’s multi-viewpoint approach minimizes these issues by piecing together visual information from different perspectives, resulting in a more accurate and detailed reconstruction. This capability is especially crucial when dealing with non-cuboid or irregularly shaped rooms, where accurate detection and connection of segments from various viewpoints can make a significant difference. Our experimental results demonstrate this advantage, as the multi-viewpoint tests achieved a PE of 0.77\% and a CE of 0.68\%, marking a substantial improvement over competing methods.

\subsection{Handling of Occlusions and Irregular Surfaces}
A key strength of SKBD is its robust handling of occlusions and irregular room layouts, as highlighted by the qualitative results (Figure \ref{fig:comparisontest}). Traditional methods, such as MC-FCN and HGC, tend to perform poorly when faced with rooms featuring occlusions, clutter, or non-cuboid geometries, leading to higher error rates. These methods are often designed with cuboid rooms in mind, and as such, struggle when applied to more complex layouts. In contrast, SKBD excels in such scenarios due to its focus on segment knotting and boundary detection. By accurately detecting boundaries and associating segments, SKBD can reconstruct irregularly shaped rooms with far greater precision, even when certain areas are obscured or contain complex surfaces. This capacity to handle occlusions and unusual geometries significantly enhances its performance compared to conventional methods.

\subsection{Generalizability Across Datasets}
The generalizability of SKBD is another key advantage, as demonstrated by its consistent performance across different datasets. In addition to excelling on the Structured3D dataset, SKBD also performs well on the LSUN dataset, despite the latter’s more challenging image quality and adherence to the Manhattan World Assumption. While the LSUN dataset presents greater difficulty due to lower image quality and more complex environments, SKBD still outperforms MC-FCN and HGC, though with slightly higher error rates. This demonstrates that SKBD is not limited to simulated, high-quality datasets but is also capable of effectively handling real-world data with more variation in structure and visual quality. The PE and CE on the LSUN dataset, at 1.13\% and 1.06\%, respectively, further illustrate SKBD’s superior performance across diverse environments, maintaining its edge over competing approaches.

\subsection{Limitations}
While SKBD has shown significant improvements over existing methods, there are several limitations that need to be addressed. First, the method’s reliance on high-quality input images limits its applicability in scenarios with lower resolution or noisy data, such as images taken in poor lighting conditions or with low-quality cameras. The accuracy of the reconstructed room layouts decreases noticeably when the image quality is suboptimal, which can affect the overall performance of the system, particularly on real-world datasets where ideal conditions cannot always be guaranteed.

Second, SKBD's multi-viewpoint strategy, while advantageous in many cases, may not be as effective in environments with extreme occlusions or significant clutter. In such situations, the algorithm may struggle to detect and connect segments accurately, leading to potential inaccuracies in the final reconstruction. 

Additionally, SKBD is currently optimized for non-cuboid room geometries, which means that its performance in cuboid rooms may not be fully realized. This specialization could limit its application in simpler scenarios where cuboid layouts are more common.

Lastly, while the algorithm performs well under the Manhattan World Assumption, many real-world environments do not conform to this constraint. This limitation may restrict SKBD's applicability in more complex spatial arrangements found in actual buildings.
