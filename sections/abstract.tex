\begin{abstract}
Image-based room layout reconstruction aims to predict the spatial structure of a room from a small number of images. Most previous papers relied on heavy assumptions of the layout, i.e. the room having a cuboid shape, a single ceiling and a single floor. This creates a need for a more general approach which can be used on a wider range of different room types. Our work focuses on improving the predictions for rooms that does not necessarily satisfy the restrictions of previous approaches. This is achieved with the help of our Segment Knotting Boundary Detection (SKBD).

The main novelty of this method is that it detects the connection of segments without heavily relying on their proximity. The process takes several steps, merging the representations into segments, associating segments into knots and connecting the knots. The output of the SKBD makes it possible to create a layout guess and reconstruct the floor map. The experimental results demonstrate that our system vastly outperforms current state-of-art methods on datasets consisting of rooms with unusual layout, but it also performs similarly to them on regular datasets.
\end{abstract}